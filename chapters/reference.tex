\chapter{Références} 
\label{chap:Références} 


\noindent

[1] Bureau Business France d’ALGER , "Point sur l’agriculture en 2020 - Algérie" , Presse République 
\indent
Française Liberté Égalité Fraternité , 10 Janvier 2021 (\url{https://ume.la/1gGgfr}).
  
[2] Une serre agricole (\url{https://ume.la/vJQBl6}).

[3] Exemple de calendrier de production de la tomate , Anne Weill et Jean Duval,Module 4, 
 \indent
 Production de transplants et de légumes en serres - Chapitre 7, « Cultures en serre », manuscrit du \indent Guide de gestion globale de la ferme maraîchère biologique et
  diversifiée, page 2.
  
[4] Anne Weill et Jean Duval,Module 4, Production de transplants et de légumes en serres - Chapitre \indent 7, « Cultures en serre », manuscrit du Guide de gestion globale de la ferme maraîchère biologique \indent et diversifiée, page 4.
 
[5] Article sur Nutrition de la tomate "Principes agronomiques de la tomate" , Knowl-
  edge grows \indent yara france,paraghraph 2. (\url{https://ume.la/WXiNWQ}).
 
[6] Lumiere serres (\url{https://ume.la/EKMJVZ}).

[7]  Antoine de France Serres, Article « La culture de tomates de A à Z » (\url{https://ume.la/my8elh}).

[8] OierSuamme, article "Tomate sous tunnel",janvier 2013 (\url{https://ume.la/pOTVXZ}).

[9] irrigation goutte-a-goutte (\url{https://ume.la/MAUKYO}).

[10] Alexander S. Gillis, “What Is IoT (Internet of Things) and How Does It Work?” IoT Agenda, \indent TechTarget,11 Feb 2020. (\url{https://ume.la/P64Xw3}).

[11] Article written by shivalibhadaniya and translated by Acervo Lima,”Caractéristiques de l’Internet \indent des objets”,StackLima,juillet 5, 2022.
(\url{https://ume.la/YhvW7R}).

[12]Capteur d'humidité Température numérique DHT22 .  (\url{https://ume.la/ORgym6}))

[13] CAPTEUR DE LUMIÈRE TSL2561. (\url{https://ume.la/y6kRpZ})

[14] Capteur d'humidité du sol. (\url{https://ume.la/dI7FZk})

[15] Arduino uno . (\url{https://ume.la/HgTH8Q})

[16] Relais . (\url{https://ume.la/GPYhEA})

[17] CONTACTER BE101 NP29 . (\url{https://ume.la/1hCdDU})

[18] Article "Qu'est ce que php ?" (\url{https://ume.la/SyzRps})

[19] Bjarne Stroustrup, Article ‘‘C++’’, 1983 (dernière révision en 2008)
(\url{https://ume.la/9ZNoPp})

[20] Agence digital PAPPLEWEB, Article “ Définition de LARAVEL (\url{https://ume.la/9FLSx8})

[21] Article ‘‘Arduino Integrated Development Environment (IDE) v1 (\url{https://ume.la/mNdBiA})

[22] Wikipedia , Article "Visual Studio Code" 25 février 2023,(\url{https://ume.la/hjnHpP})

[23] Article "Qu’est-ce que Visual Paradigm Online ?", (\url{https://ume.la/KsYyTL})

[24] Wikipedia ,Article "Fritzing" , (\url{https://ume.la/eGOVGe})

[25] Article “MySQL”, (\url{https://ume.la/zieO2y})

[26] JDN,Article ‘‘API (interface de programmation) , (\url{https://ume.la/RGPDkI})

[27] wikipedia ,Article "Modèle-vue-contrôleur",(\url{https://ume.la/muacZ6})
