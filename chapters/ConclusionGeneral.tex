\chapter{Conclusion Géneral} \label{chap:Conclusion Géneral}
L`internet des objets (IoT) est une technologie révolutionnaire qui permet à des objets physiques de collecter, de communiquer et d'interagir avec le monde numérique. Il s'agit d'un réseau de dispositifs connectés qui peuvent échanger des données et effectuer des actions autonomes en fonction de ces données. L'IoT repose sur l'utilisation de capteurs, de réseaux pour la communication et de logiciels pour permettre la connectivité entre les objets et mettre l'échange d'informations facile.
\newline

Notre projet vise l’agriculture intelligente en général et les SI particulièrement. Le projet a mis en place un système intelligent de contrôler et automatisé les actions des cultures afin de notre objectifs étaient d’améliorer la productivité, l'efficacité et la durabilité de l'agriculture et comme présenté au introduction général ‘‘objectifs ’’ :
\\
1-  Une étude a été réalisée sur la production de la tomate industrielle et des paramètres a été ressorties pour l’agriculture du tomate sous serres.
\\
2-  Ce système répond à des exigences de culture sous serres et se base d’une parte sur Hardware qui est représenté par des capteurs dans l’intention de collecter les données à travers l’Arduino .
\\
3-  Des taches automatiser avec certain dispositifs électroniques après le collecte et le traitement des données sous des conditions a fin de l’envoyer à l’application serveur.
\\
4-  D’autre part, le côté Software, ceci nous a permis de réaliser une maquette basée sur Arduino et d’un système Client Serveur interfacer d’une parte a cette maquette et d’autre parte a l’utilisateur sous forme Web. Le système récupéré les données pour l’objectif de les exploiter au l’utilisateur, de les stocker afin d’analyser futurement et pour suivre l’évolution de la saison. 
\\
5-  Notre système a été tester sur plusieurs conditions et donne satisfaction. Le taux de réalisation de ce prototype est d’enivrant 80 \%. Vu que seulement quelque capteurs ont été utilisé et non pas la totalité pour une agriculture sous serre . 

\section*{Perspectives}
 En guise de perspectives, nous comptons en premier, d’intégrer plus des dispositifs électriques pour donner à l’utilisateur un contrôle complet de son environnement. Puis, nous aura développer les points suivants futurement :
 \\
 - Renforcer la maquette par des capteur qui manquant comme chauffage , AC ... 
 \\
 - Faire évoluer le système pour qu'il puisse serve plusieurs personnes.
 \\
 - Intégrer des solutions basées sur l'apprentissage automatique pour analyser et traitement des données.
 - Intégration des énergies renouvelables.
 \\
 \newline
 
\noindent
\textbf{En conclusion}, ce projet final nous a été très bénéfique, et nous avons acquis diverses connaissances et compétences en matière de : synthétiser des documents scientifiques, concevoir et modéliser des projets informatiques, notamment programmer avec différents outils et plateformes. Nous espérons que ces compétences nous aideront à l'avenir, et nous espérons que ce travail sera une référence pour ceux qui veulent développer des systèmes dans ce domaine avec d'autres fonctionnalités. 
manière automatisée.
