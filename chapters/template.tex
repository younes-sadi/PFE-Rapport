\chapter{Template Items} \label{chap:template}

This part contains the typographical elements of the template, to be used in writing your Master's thesis. A course on scientific writing using \LaTeX{} is available at:
\url{https://drive.google.com/file/d/1coBxyvq-XRw5Sr3GO-VDJhYsPSLOQpRD/}
\\
\\
\textcolor{red}{
	This chapter aims to give you examples of the template. You must absolutely remove it during the final version of the thesis.
}

\section{Title - Level 2} \label{sec:example}
\subsection{Title - Level 3}
\subsubsection{Title - Level 4}

\section*{Title - Level 2 (Unnumbered)}
\subsection*{Title - Level 3 (Unnumbered)}
\subsubsection*{Title - Level 4 (Unnumbered)}

\section{Lists of Items}
This is normal text. followed by a list of items:

\firmlist
\begin{itemize}
	\item Item 1	
	\item Item 2
	\begin{itemize}
		\item Item A
		\item Item B
		\begin{itemize}
			\item Item I
			\item Item II
			\item ...
		\end{itemize}
	\end{itemize}
\end{itemize}

\noindent
And here is an enumerated list of items:

\begin{enumerate}
	\item Item 1	
	\item Item 2
	\begin{enumerate}
		\item Item A
		\item Item B
		\begin{enumerate}
			\item Item I
			\item Item II
			\item ...
		\end{enumerate}
	\end{enumerate}
\end{enumerate}

\section{Figures, Tables and Algorithms}
You can define several types of floating elements: Figures, tables, and algorithms.

\begin{figure}[h]
	\centering
	\includegraphics[width=2.5cm]{front/logo-uc2}
	\caption{An example of figures}
	\label{fig:example} % label must always be after caption
\end{figure}

\begin{table}[h]
	\centering
	\caption{An example of tables}
	\begin{tabular}{|L{5.1cm}C{5.1cm}R{5cm}|}  % sum=15.2cm (L=Left, C=Center, R=Right)
	\hline
	Colonne 1 & Colonne 2 	 & Colonne 3 \\
	Ligne 1   & Ligne 1   	 & Ligne 1 	 \\
	Ligne 2   & Ligne 2 	 & Ligne 2 	 \\
	...	      & ... 		 & ... 		 \\
	...		  & ... 		 & ... 		 \\
	...		  & ... 		 & ... 		 \\
	\hline
\end{tabular}
	\label{tab:example} % label must always be after caption
\end{table}

\begin{algorithm}[h] 
	\selectlanguage{french}
	\begin{algorithmic}[1]
	\REQUIRE $i\in \mathbb{N}$
	\STATE $i\gets 10$
	\IF {$i\geq 5$} 
		\STATE $i\gets i-1$
	\ELSE
		\IF {$i\leq 3$}
			\STATE $i\gets i+2$
		\ENDIF
	\ENDIF 
\end{algorithmic}

%Documentation: https://mirror.marwan.ma/ctan/macros/latex/contrib/algorithms/algorithms.pdf

%\STATE <text>
%\IF{<condition>} \STATE {<text>} \ELSE \STATE{<text>} \ENDIF
%\IF{<condition>} \STATE {<text>} \ELSIF{<condition>} \STATE{<text>} \ENDIF
%\FOR{<condition>} \STATE {<text>} \ENDFOR
%\FOR{<condition> \TO <condition> } \STATE {<text>} \ENDFOR
%\FORALL{<condition>} \STATE{<text>} \ENDFOR
%\WHILE{<condition>} \STATE{<text>} \ENDWHILE
%\REPEAT \STATE{<text>} \UNTIL{<condition>}
%\LOOP \STATE{<text>} \ENDLOOP
%\REQUIRE <text>
%\ENSURE <text>
%\RETURN <text>
%\PRINT <text>
%\COMMENT{<text>}
%\AND, \OR, \XOR, \NOT, \TO, \TRUE, \FALSE
	\caption{An example of algorithms}
	\label{algo:example} % label must always be after caption
\end{algorithm}

\section{Cross-Referencing}

By using labels, it is possible to reference different elements of the document. As examples, Chapter \ref{chap:template}, Section \ref{sec:example}, Figure \ref{fig:example}, Table \ref{tab:example}, Algorithm \ref{algo:example} and Definition \ref{def:example}.


\begin{definition} \label{def:example}
	(Title of the definition)\\
	An example of definitions, $E=mc^2$...
\end{definition}

In addition to definitions, you can use theorems, proofs, remarks, notations, lemmas, or propositions.

\section{Source Codes}

You can also introduce source codes, like the following example which is written in Java language (The syntax highlighting can be customized in the file \texttt{"/macros.tex"}):

\begin{sourcecode}{A.java}
	\lstinputlisting[language=Java]{listings/A.java}
\end{sourcecode}
\begin{sourcecode}{A.py}
	\lstinputlisting[language=Python]{listings/A.py}
\end{sourcecode}

