\begin{abstractAr}
  المزرعة الذكية في البيوت البلاستكية هو نظام زراعي متقدم يدمج التقنيات الذكية وتحليل البيانات لتحسين نمو المحاصيل والحصاد في بيئة مراقبة. يستخدم النظام شبكة من الحساسات وأجهزة المراقبة لجمع البيانات عن مختلف المعلمات البيئية مثل درجة الحرارة والرطوبة وشدة الإضاءة. يتم تحليل هذه البيانات باستخدام خوارزميات التعلم الآلي لتوفير إدراكات حول أنماط نمو النباتات، وتوقعات الحصاد.

	تم تصميم المزرعة الذكية لتوفير ردود فعل في الوقت الحقيقي للمزارعين ومديري البيوت الزراعية، مما يتيح لهم اتخاذ قرارات مدروسة بشأن إدارة المحاصيل. كما يوفر النظام الوصول عن بعد لمراقبة والتحكم في البيئة، مثل ضبط درجة الحرارة والرطوبة، وضمان الظروف المثلى للنمو.
	
	تتميز تكامل المزرعة الذكية بالعديد من الفوائد، بما في ذلك زيادة الكفاءة، وتقليل استهلاك الموارد، وتحسين حصاد المحاصيل. يتيح النظام للمزارعين تحسين عملياتهم لإنتاج المزيد من الغذاء بشكل مستدام، مع تقليل التكاليف وتقليل الأثر على البيئة.
	
	بشكل عام، يمثل المزرعة الذكية خطوة مهمة في الزراعة المستدامة، وإمكاناته لتحول الصناعة هامة. من خلال الاستفادة من التكنولوجيا المتقدمة.
	
	
	
\end{abstractAr}

